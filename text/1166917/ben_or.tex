\documentclass[12pt, a4paper]{article}
\usepackage{hyperref}
\usepackage{amsthm}
\usepackage{amsmath}
\usepackage{mathtools}
\usepackage[
    bindingoffset=-15pt,
	voffset=0pt, 
	hoffset=10pt, 
	textwidth=500pt, 
    textheight=650pt, 
	marginparwidth=20pt,
	footskip=60pt
]{geometry}
\usepackage{algorithm}
\usepackage[noend]{algpseudocode}

\headheight=28pt



\begin{document}

\newtheorem{lemma}{Lemma}
\newtheorem{theorem}{Theorem}

\section*{Ben-Or asynchronous Byzantine agreement algorithm}
\subsection*{Introduction}
The idea of the Byzantine agreement problem is to have a system of $N$ processes, $t$ of which are faulty. Each correct process starts with either 0 or 1, and after running the algorithm, all correct processes must end up with either 0 or 1 based on the following rules:
\begin{itemize}
    \item All correct processes end up with the same value.
    \item If all correct processes start with the same value $v$, then all correct processes must end with $v$.
    \item All correct processes must terminate (with probability 1).
\end{itemize}
Ben-Or algorithm allows the network to be asynchronous and does not restrict the malevolence of faulty processes. It works under assumption $t<\frac{N}{5}$.
\subsection*{Protocol}
Let $N$ - number of all processes, $t$ - number of faulty processes, $t<\frac{N}{5}$\\
Let's bring up the pseudo code\cite{Ben83}:
\begin{flushleft}
\textbf{Process} P: Initial value $x_p$.\\
\textbf{step 1}: Set $r\coloneqq 1$.\\
\textbf{step 0}: Send the message (1, $r$, $x_p$) to all the
processes.\\
\textbf{step 2}: Wait till messages of type (1, $r$, *) are received from $N-t$ processes. If more than $(N-t)/2$ messages have the same value $v$, then send the message (2, $r$, $v$, D) to all processes. Else send the message (2, $r$, ?) to all processes.\\
\textbf{step 3}: Wait till messages of type (2, $r$, *) arrive from $N-t$ processes.\\
(a) If there are at least $t+1$ D-messages (2, $r$, $v$, D), then set $x_p \coloneqq v$.\\
(b) If there are more than $(N + t)/2$ D-messages then \textbf{decide} $v$.\\
(c) Else set $x_p=$ 1 or 0 each with probability $\frac{1}{2}$.\\
\textbf{step 4}: Set $r \coloneqq r + 1$ and go to step 1.
\end{flushleft}
Processes locally keep track of current round number, and store current binary value.\\
Within each round they go through 2 stages, messages from each round from now on will be called type 1 or 2 messages.
\subsection*{Correctness}
Let's start with proving the following lemma:
\begin{lemma} 
If all correct processes start with the value $v$, then within one round they will all decide $v$.
\end{lemma}
\begin{proof}
Each process broadcasts message $(1,\ 1,\ v)$.\\
Process then waits for $N-t$ messages.\\
At most $t$ messages came from faulty processes, which means that at least $N-2\cdot t$ messages are correct. Because $N-2\cdot t>\frac{N+t}{2}$, all the correct processes send message $(2,\ 1,\ v,\ D)$.\\
Among $N-t$ accepted type 2 messages at most $t$ are incorrect, which means that step 3(b) will not execute.\\
Again, because at least $N-2\cdot t$ of type 2 messages are correct, then more than $\frac{N+t}{2}$ of them will have the same value (v). That means that every process will \textbf{decide $v$} this round.
\end{proof}
\begin{lemma}
If process sets $v$ in step 3(a) in round $r$, then it can't set $\neg v$ in the same round.
\end{lemma}
\begin{proof}
Let's say that process saw $\geq t+1$ messages  $(2,\ r,\ 0,\ D)$ and $\geq t+1$ messages $(2,\ r,\ 1,\ D)$. Let $A_x$ be equal to the number of messages $(2,\ r,\ x,\ D)$ originated from correct processes with $x=0,1$. Of course $A_0+A_1\geq t+2$.\\
Every process responsible for $A_x$ saw more than $\frac{N-t}{2}$ messages $(1,\ r,\ x)$ from correct process. But that means that there are more than $N-t$ correct processes, which is a contradiction.
\end{proof}
Then let's look at the next lemma, which states that the processes will decide with at most 2 round window.
\begin{lemma}
If for some round r, some correct process decides $v$ in step 3(b), then all other correct processes will decide $v$ within the next round.
\end{lemma}
\begin{proof}
Let $P$ be the process that has decided on $v$.
In order to \textbf{decide $v$}, $P$ must have received more than $\frac{N+t}{2}$ messages $(2,\ r,\ \_,\ D)$. That means that more than $\frac{N-t}{2}$ correct processes sent that message. Let $A_x$ be the number of $(2,\ r,\ x,\ D)$ messages from correct processes for $x=0,1$.\\
Because $P$ has decided $v$, it follows that $A_{\neg v}\leq t$. On top of that $A_v+A_{\neg v}>\frac{N-t}{2}$.\\
Simple calculations show that $A_v\geq t+1$. It means that every other process will set $v$ in step 3(a). From \textit{Lemma 2} we know that it will be the only value that they will set.\\
It follows that in the next round every correct process will start with $v$, so \textit{Lemma 1} can be applied.
\end{proof}
\subsection*{Time complexity}
In each round processes have probability of at least $2^{-(N-t)+1}$ of all setting the same value, which using \textit{Lemma 1} means that the run algorithm would end next round. That means that expected number of rounds is bounded by $O(2^{N-t})$.\\
The following theorem shows tighter bound under stricter assumptions:
\begin{theorem}\cite{Ben83}
If $t=O(\sqrt{N})$ then the expected number of rounds to reach agreement in this protocol is constant, i.e., it does not depend on N. 
\end{theorem}
\begin{thebibliography}{9}
\bibitem{Ben83} Michael Ben-Or. Another Advantage of Free Choice: Completely Asynchronous Agreement Protocols. In \textit{Proceedings of the second annual ACM symposium on Principles of distributed computing}, pages 27-30. ACM, 1983, doi: 10.1145/800221.806707.
\end{thebibliography}

\end{document}