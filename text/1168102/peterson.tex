\documentclass{article}

\usepackage[english]{babel}
\usepackage{filecontents}
\usepackage[letterpaper,top=2cm,bottom=2cm,left=3cm,right=3cm,marginparwidth=1.75cm]{geometry}

\usepackage{amsmath}
\usepackage{csquotes}
\usepackage{graphicx}
\usepackage{amsthm}

\usepackage[colorlinks=true, allcolors=blue]{hyperref}
\begin{filecontents}{\jobname.bib}
@article{10.1145/69622.357194,
author = {Peterson, Gary L.},
title = {An O(nLog n) Unidirectional Algorithm for the Circular Extrema Problem},
year = {1982},
issue_date = {Oct. 1982},
publisher = {Association for Computing Machinery},
address = {New York, NY, USA},
volume = {4},
number = {4},
issn = {0164-0925},
url = {https://doi.org/10.1145/69622.357194},
doi = {10.1145/69622.357194},
journal = {ACM Trans. Program. Lang. Syst.},
month = {oct},
pages = {758–762},
numpages = {5}
}
\end{filecontents}

\usepackage{natbib}
\usepackage{cleveref}
\newtheorem{lemma}{Lemma}
\newtheorem{theorem}{Theorem}
\begin{document}
\section*{Leader election in an directed ring}

Given $n$ processes in a directed ring with synchronous communication, communicating only with message passing to its neighbors, the circular extrema-finding (or election) problem is to select a maximum process. Each process has a unique value in a set with a total order. These values may be transmitted and compared. All processes are identical (except for their value) i.e run the same algorithm and $n$, the number of processes, is not initially known.

Here we describe an algorithm presented by \cite{10.1145/69622.357194}, which achieves the above and uses at most $2 n \lfloor\log n\rfloor + n$ messages in the worst case.
\subsection*{The algorithm}
\subsubsection*{Definitions and assumptions}
The algorithm will consist of electoral rounds.

We will distinguish 4 states a node can be in:
\begin{description}

\item[\textit{active}] --- node is taking part in this electoral round
\item[\textit{relay}] --- node does not participate in election anymore and only passes messages
\item[\textit{non-leader}] --- node is not the leader
\item[\textit{leader}] --- node is the leader

\end{description}

Messages sent will consist of pairs of "value" and "status". Values will be from the set of unique identifiers of the nodes participating, while status can be either \textit{active} or \textit{leader}.

\subsubsection*{Description}

Assume all messages are passed to the right. All processes are initially \textit{active}. \textit{Active} processes operate in phases. \textit{Relay} processes in one phase just pass two at most messages they receive. The primary goal is to have the number of active processes be cut at least in haft during each phase. Each process starts with its own value as its temporary identifier (\text{tid}). During a phase, each \textit{active} process receives the \textit{tid} of its nearest \textit{active} neighbor to the left and that neighbor's nearest left \textit{active} neighbor's \textit{tid}. If the first process sees that its left \textit{active} neighbor has the largest of the three \textid{tid}s, then it sets its \textit{tid} to that value and starts the next phase. Otherwise, it becomes a \textit{relay} process.

\subsubsection*{Procedures}

Each round starts with each \textit{active} sending message ($tid$, \textit{active}) to right neighbour. Afterwards, each \textit{active} awaits to receive a message from left.

Let's denote message received from left as $(m1.ntid, m1.status)$. We need to \textit{check} whether $m1.status$ is \textit{leader} or $m1.ntid$ is equal to node's unique identifier. In first case we need set our state to \textit{non-leader} and pass the message to right. In second case we need to declare this node as leader: set state to \textit{leader} and send message with $(i, \textit{leader})$ to right. If \textit{check} is failed then we wait for one more message from left as $(m2.nntid, m2.status)$. Do the same \textit{check} as previously, and if it's failed then there are two cases. In first case $m1.ntid \geq tid \land m1.ntid \geq m2.nntid$ we assign $m1.ntid$ to node's $tid$, otherwise we set the node state to \textit{relay}.

Nodes that become \textit{non-leader} are terminated. \textit{Leader} finishes when it receives message with status \textit{leader}.

\textit{Relay} waits for at most two messages from left processed one by one, for each message do \textit{check} and if it's failed just pass the message to the right.

\subsection*{Correctness}
The correctness of the algorithm follows from two facts: during any phase there is exactly one process whose \textit{tid} is the maximum value, and no process except the maximum will ever see its own initial value. The first fact comes by noting that the process whose \textit{tid} is the maximum value is by definition larger than its two \textit{active} neighbors, so the \textit{active} process to its right will have its \textit{tid} equal to the maximum on the next phase. The second fact follows by first noting that, for any process, its value is either the \textit{tid} value of the nearest \textit{active} process to its right, or its value no longer is being propagated. It therefore follows that in order for some nonmaximum value of a process to cycle around the ring, the largest value must already have passed the process (lest the \textit{active} process with \textit{tid} equal to the maximum prevents the passage of the lower value), but then the smaller value is no longer being propagated since the \textit{tid} value to the right of the process is now the maximum.
\subsection*{Complexity analysis}
It is very simple to see that at most $2n \log n + n$ messages are sent. First one
notes that there can be at most $\lfloor\log n\rfloor$ phases to reach one \textit{active} process. This follows from the fact that any \textit{active} process survives a phase only if the \textit{active} process to its left does not survive that phase. Hence of $m$ \textit{active} processes during a phase, at most $\lfloor\frac{m}{2}\rfloor$ will survive for $m > 1$. During a phase every process sends (and receives) two messages, and on the last phase the lone \textit{active} process sends one message which is relayed around to the maximum for a total of $2n \log n + n$.
\bibliographystyle{plainnat}
\bibliography{\jobname} 
\end{document}