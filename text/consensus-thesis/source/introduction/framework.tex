To simulate chosen protocols in a practical environment we implement them in the Go Programming Language \cite{DK15}. The implementations use Distributed Framework \cite{Tur+22} and are incorporated into its repository. We are going to describe how the framework works before heading into details of our implementations. 

The crucial part of the framework is a library that enables its users to simulate distributed protocols, which are usually run on multiple processors, on a single processor by using Go language-specific features like simple in use and efficient goroutines and channels. Any user of the framework, besides adding his protocol, needs to only care about setting up a relevant network and a type of communication. Let us list the most important elements of the library.

\begin{description}
    \item[Nodes] Each simulated processor is represented by an object of type \texttt{Node}. In particular, nodes provide \texttt{SendMessage()} and \texttt{ReceiveMessage()} methods used for communication. Moreover, they allow to conveniently share relevant data between rounds and with external observer using \texttt{SetState()} and \texttt{GetState()} methods.
    
    \item[Graph] The nodes can be connected in multiple ways with the use of included graph building functions family. In our code, we are going to use \texttt{BuildCompleteGraphWithLoops()} which accordingly to its name constructs and returns a list of nodes connected by channels in a complete graph with loops. It also returns an instance of \texttt{Synchronizer} corresponding to the created graph.

    \item[Synchronizer] Type \texttt{Synchronizer} provides a mechanism of synchronization. Its \texttt{Synchronize()} method initialize a process. Then, at the beginning of each round nodes need to call \texttt{StartProcessing()} method and \texttt{FinishProcessing()} method at the end of it. The synchronizer assures that no two nodes are in different rounds, by suspending nodes with \texttt{StartProcessing()} until every node calls \texttt{FinishProcessing()} in the previous round (if only such exists). The synchronizer also counts the number of rounds and messages sent within all the rounds. One can obtain these statistics using \texttt{GetStats()} method.
\end{description}



Besides the library, the framework contains numerous implementations of protocols. These are mainly for the leader election problem on different networks or for some graph problems (e.g. independent set, minimum-weight spanning tree). It also includes the Ben-Or's protocol implementation. As mentioned earlier, in this work we contribute to the Distributed Framework project by opening several accepted pull requests in which we complete items as follows \cite{Pac22a, Pac22b, Pac22c, Pac22d}.

\begin{enumerate}
    \item Organizing and incorporating a unified naming convention for the files in the repository.
    \item Adding the single-bit message protocol.
    \item Adding the phase king protocol.
    \item Refactoring the existing Ben-Or's protocol and unifying it with other distributed consensus protocols.
    \item Adding the faulty behavior strategies for all the implemented distributed consensus protocols.
    \item Adding the text of current thesis with the script for reproducing the presented figures.
\end{enumerate}