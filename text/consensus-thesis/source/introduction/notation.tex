Let us introduce the notation that will be consistently used throughout this paper. Just like in \Cref{in:def:consensus}, by $n$ we denote the total number of processors and by $t$ the number of the faulty processors among them. Every processor has its id stored as $i$. As a reminder, an id is a number between $1$ and $n$ which is unique for each processor. Moreover, every faulty processor has its \emph{faulty id} stored as $j$. The faulty id is also unique for each processor, but it is a number between $1$ and $t$. Each proposed protocol repeats a certain procedure and a single invocation of such procedure will be referred to as \emph{a phase}. During the phase, processors, usually more than once, synchronize to exchange their messages. Such an event is called \emph{a message exchange round}. In pseudocodes of the protocols, one can find \textsc{Broadcast($x$)} and \textsc{Receive()} functions. Unless stated otherwise, they work as follows. \textsc{Broadcast($x$)} is responsible for sending value $x$ to all processors including the processor from which it is executed.
\textsc{Receive()} is used to receive an array of all messages sent in the current message exchange round to the processor executing the function. Messages in the array are ordered by the senders' ids. There is also a \textsc{Send($x, i$)} function sending value $x$ to the processor with the id $i$.
Additionally, when considering the faulty behavior strategies, we use \textsc{PeekV()} function which returns an array of all processors' $V$ values. If some processor does not define $V$, we treat it as $nil$ value. Values in the array are ordered by the ids of the processors they come from.